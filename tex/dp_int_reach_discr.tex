\documentclass[10pt, a4paper]{article}
\usepackage{triada}

\graphicspath{{eps/}{png/}}

\renewcommand{\ell}{\mathcal{E}}

\usepackage{delim}

                           
\begin{document}
\thispagestyle{empty}

\begin{center}
\ \vspace{-3cm}

\includegraphics[width=0.5\textwidth]{msu.eps}\\
{\scshape Московский государственный университет имени М.~В.~Ломоносова}\\
Факультет вычислительной математики и кибернетики\\
Кафедра системного анализа

%\vspace{5cm}
\vfill
{\LARGE Курсовая работа}

\vspace{1cm}

{\Huge\bfseries <<Изучение динамических систем>>}
\end{center}

\vspace{2cm}

\begin{flushright}
  \large
  \textit{Студент 315 группы}\\
  Д.~И.~Степенский

  \vspace{5mm}

  \textit{Руководитель практикума}\\
  к.ф.-м.н., ассистент П.~А.~Точилин
\end{flushright}

\vfill

\begin{center}
Москва, 2012
\end{center}

\newpage

\tableofcontents

\newpage

\section{Постановка задачи}
Рассматривается следующая дискретная система:
\begin{equation}\label{statement_system} \begin{cases}
{x}(k+1) = A(k)x(k)+B(k)u(k),\quad k_0\leqslant k\leqslant k_1],\\
x(k_0)\in \ell(x_0,X_0), \\
u(k) \in \ell(q(k),Q(k)),\\
u(k)\in \mathbb{R}^m, x(k)\in \mathbb{R}^n.
\end{cases} \end{equation}  

Требуется построить внутреннюю эллипсоидальную оценку множества достижимости данной системы.

\section{Теоретическая часть}
\subsection{Множество достижимости}
Вначале будем считать, что матрицы $Q(i)$ и $A(i)$ несингулярны, т.е. имеют полный ранг. Случай сингулярных матриц будет разобран позднее.
\begin{df}
\textit{Эллипсоидом} $\ell(q,Q), Q\in\mathbb{R}^{n\times n}$ в $n$-мерном пространстве назовём множество точек
\[ \ell(q,Q) = \left\{ x\colon (x-q)Q^{-1}(x-q)' \leqslant 1 \right\}. \]
Здесь $q$ --- \textit{центр} эллипсоида, а $Q$ --- \textit{матрица} эллипсоида. 
\end{df}
Эллипсоид назовем \textit{вырожденным}, если его матрица имеет неполный ранг, или, что равносильно, $\det Q = 0$.
\begin{df}
\textit{Множеством достижимости} дискретной системы, стартующей в момент $k_0$ из точки $x_0$ в момент времени $k$ называется следующее множество:
\[ \mathcal{X}(k,k_0,x_0) = \mathcal{X}[k] = \left\{ x \middle| \exists u(s), k_0\leqslant s\leqslant k\colon x(k|k_0,x_0)=x \right\}.\]
\end{df}
Понятие множества достижимости обобщается на случай произвольной стартовой точки из множества $\mathcal{X}_0$:
\[ \mathcal{X}(k,k_0,\mathcal{X}_0) = \left\{ x \middle| \exists u(s), k_0\leqslant s\leqslant k\colon x(k|k_0,x_0)\in \mathcal{X} \right\} = \bigcup\limits_{x_0\in\mathcal{X}_0}\mathcal{X}(k,k_0,x_0).\]
\begin{theorem}
Решением системы \eqref{statement_system} является 
\begin{equation}\label{solution}
	x(k) = \Phi(k,k_0)x_0 + \sum\limits_{i=k_0}^{k-1}\Phi(k,i+1)B(i)u(i),
\end{equation} 
где $\Phi(k,p)$ --- переходная матрица:
\begin{equation}\label{solution}
	\Phi(k+1,p) = A(k)\Phi(k,p),\ \Phi(p,p)=I.
\end{equation}
\end{theorem}
\begin{proof}
	Докажем по индукции. Для $k=k_0+1$ очевидно. Пусть для $k$ показано, тогда для $k+1$
	проверим подстановкой в \eqref{statement_system}:
	\begin{gather*}
		{x}(k+1) = \Phi(k+1,k_0)x_0 + \sum\limits_{i=k_0}^{k+1-1}\Phi(k+1,i+1)B(i)u(i) = 
		\\ = A(k)\Phi(k,k_0)x_0 + A(k)\sum\limits_{i=k_0}^{k+1-2}\Phi(k,i+1)B(i)u(i) + 
		\Phi(k+1,k+1)B(k)u(k) = \\ = A(k)x(k) + B(k)u(k).
	\end{gather*}
\end{proof}
\begin{imp}
	Используя теорему, множество достижимости системы \eqref{statement_system} можно записать следующим образом:
	\begin{gather}
		\notag \mathcal{X}(k,k_0,\ell(x_0,X_0)) = q_c(k) + \Phi(k,k_0)\ell(0,X_0) + 
			\sum\limits_{i=k_0}^{k-1}\Phi(k,i+1)B(i)\ell(0,Q(i)) = \\
		= q(k) + \Phi(k,k_0)\ell(0,X_0) + 
			\sum\limits_{i=k_0}^{k-1}\Phi(k,i+1)\ell(0,B(i)Q(i)B(i)'),
	\end{gather}
	где \begin{gather}
		q_c(k) = \Phi(k,k_0)x_0 + \sum\limits_{i=k_0}^{k-1}\Phi(k,i+1)B(i)q(i).
	\end{gather}
\end{imp}
Здесь знак <<$+$>> означает сумму множеств по Минковскому: $A+B= \left\{x+y \middle| x\in A,\, y\in B\right\}$, а умножение множества на матрицу производится по формуле $\Phi A = \left\{ \Phi x\middle| x\in A \right\}.$ Кроме того, использовалось свойство эллипса $\Phi\ell(x,X) = \ell(\Phi x,\Phi X\Phi')$.
\subsection{Внутренние эллипсоидальные оценки}
Если умножение на невырожденную матрицу не выводит из класса эллипсоидов, то сумма двух эллипсоидов, вообще говоря, этому классу не принадлежит, поэтому множество достижимости конкретным эллипсоидом записать не получится. Однако возможно эллипсоидальное оценивание этого множества.

Для начала найдем опорную функцию множества достижимости, пользуясь свойством её линейности относительно сумм по Минковскому, а так же тем, что опорная функция эллипсоида $\sufu{l}{\ell(x,X)}$ равна $\scalar{l}{x}+\norm{l}_X$:
\begin{gather}
\notag\sufu{l}{\mathcal{X}\left(k,k_0,\ell\left(x_0,X_0\right)\right)} = \sufu{l}{q_c\left(k\right)} + \sufu{l}{\Phi\left(k,k_0\right)\ell\left(0,X_0\right)} +\\ 
			\notag + 
			 \sum\limits_{i=k_0}^{k-1}\sufu{l}{\Phi\left(k,i+1\right)B\left(i\right)\ell\left(0,Q\left(i\right)\right)} = 
			 \scalar{l}{q_c\left(k\right)} + 
				\sufu{l}{\ell\left(0,\Phi\left(k,k_0\right)X_0\Phi\left(k,k_0\right)'\right)} + \\
			\notag +\sum\limits_{i=k_0}^{k-1} 
				\sufu{l}{
					\ell\left(0,\Phi\left(k,i+1\right)B\left(i\right)Q\left(i\right)B\left(i\right)'\Phi\left(k,i+1\right)'\right)} = 
			 \scalar{l}{q_c\left(k\right)} +\\+ \scalar{l}{\Phi\left(k,k_0\right)X_0\Phi\left(k,k_0\right)'l}^{1/2}
			  +
			\scalar{l}{\Phi\left(k,i+1\right)B\left(i\right)Q\left(i\right)B\left(i\right)'\Phi\left(k,i+1\right)'l}^{1/2}.
\end{gather}

\begin{theorem}
Эллипсоид, являющийся внутренней оценкой суммы $\ell(0,Q_1)+\dots+\ell(0,Q_n)$, имеет матрицу
\begin{equation}\label{th:ell} Q = Q_*'Q_*,\quad Q_*=\sum\limits_{1}^{n} S_kQ_k^{1/2},\ S_kS'_k=I, \end{equation}
причем для оптимальности (в смысле касания заданного множества в направлении $l$) необходима и достаточна линейная зависимость векторов $S_kQ_k^{1/2}l,\ k=1,\dots,n.$
\end{theorem}
\begin{proof}
Для доказательства внутренности оценки достаточно показать, что опорная функция эллипсоида с матрицей \eqref{th:ell} всюду меньше опорной функции суммы эллипсоидов. Сравним квадраты опорных функций и воспользуемся неравенством о средних и неравенством Коши-Буняковского:
\begin{align*}
\sufu{l}{\ell(0,Q)} &=  
\end{align*}
\end{proof}



























\end{document}