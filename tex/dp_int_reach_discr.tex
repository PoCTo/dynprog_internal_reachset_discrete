\documentclass[10pt, a4paper]{article}
\usepackage{triada}

\graphicspath{{eps/}{png/}}

\renewcommand{\ell}{\mathcal{E}}

\usepackage{delim}

                           
\begin{document}
\thispagestyle{empty}

\begin{center}
\ \vspace{-3cm}

\includegraphics[width=0.5\textwidth]{msu.eps}\\
{\scshape Московский государственный университет имени М.~В.~Ломоносова}\\
Факультет вычислительной математики и кибернетики\\
Кафедра системного анализа

%\vspace{5cm}
\vfill
{\LARGE Курсовая работа}

\vspace{1cm}

{\Huge\bfseries <<Изучение динамических систем>>}
\end{center}

\vspace{2cm}

\begin{flushright}
  \large
  \textit{Студент 315 группы}\\
  Д.~И.~Степенский

  \vspace{5mm}

  \textit{Руководитель практикума}\\
  к.ф.-м.н., ассистент П.~А.~Точилин
\end{flushright}

\vfill

\begin{center}
Москва, 2012
\end{center}

\newpage

\tableofcontents

\newpage

\section{Постановка задачи}
Рассматривается следующая дискретная система:
\begin{equation}\label{statement_system} \begin{cases}
{x}(k+1) = A(k)x(k)+B(k)u(k),\quad k_0\leqslant k\leqslant k_1],\\
x(k_0)\in \ell(x_0,X_0), \\
u(k) \in \ell(q(k),Q(k)),\\
u(k)\in \mathbb{R}^m, x(k)\in \mathbb{R}^n.
\end{cases} \end{equation}  

Требуется построить внутреннюю эллипсоидальную оценку множества достижимости данной системы.

\section{Теоретическая часть}
\begin{df}
\textit{Эллипсоидом} $\ell(q,Q), Q\in\mathbb{R}^{n\times n}$ в $n$-мерном пространстве назовём множество точек
\[ \ell(q,Q) = \left\{ x\colon (x-q)Q^{-1}(x-q)' \leqslant 1 \right\}. \]
Здесь $q$ --- \textit{центр} эллипсоида, а $Q$ --- \textit{матрица} эллипсоида. 
\end{df}
Эллипсоид назовем \textit{вырожденным}, если его матрица имеет неполный ранг, или, что равносильно, $\det Q = 0$.
\begin{df}
\textit{Множеством достижимости} дискретной системы, стартующей в момент $k_0$ из точки $x_0$ в момент времени $k$ называется следующее множество:
\[ \mathcal{X}(k,k_0,x_0) = \mathcal{X}[k] = \left\{ x \middle| \exists u(s), k_0\leqslant s\leqslant k\colon x(k|k_0,x_0)=x \right\}.\]
\end{df}
Понятие множества достижимости обобщается на случай произвольной стартовой точки из множества $\mathcal{X}_0$:
\[ \mathcal{X}(k,k_0,\mathcal{X}_0) = \left\{ x \middle| \exists u(s), k_0\leqslant s\leqslant k\colon x(k|k_0,x_0)\in \mathcal{X} \right\} = \bigcup\limits_{x_0\in\mathcal{X}_0}\mathcal{X}(k,k_0,x_0).\]
\begin{theorem}
Решением системы \ref{statement_system} является 
\begin{equation}\label{solution}
	x(k) = \Phi(k,k_0)x_0 + \sum\limits_{i=k_0}^{k-1}\Phi(k,i+1)B(k)u(k),
\end{equation} 
где $\Phi(k,p)$ --- переходная матрица:
\begin{equation}\label{solution}
	\Phi(k+1,p) = A(k)\Phi(k,p),\ \Phi(p,p)=I.
\end{equation}
\end{theorem}
\begin{proof}
	Докажем по индукции. Для $k=k_0+1$ очевидно. Пусть для $k$ показано, тогда для $k+1$
	проверим подстановкой в \ref{statement_system}:
	\begin{gather*}
		{x}(k+1) = \Phi(k+1,k_0)x_0 + \sum\limits_{i=k_0}^{k+1-1}\Phi(k+1,i+1)B(k)u(k) = 
		\\ = A(k)\Phi(k,k_0)x_0 + A(k)\sum\limits_{i=k_0}^{k+1-2}\Phi(k,i+1)B(k)u(k) + 
		\Phi(k+1,k+1)B(k)u(k) = \\ = A(k)x(k) + B(k)u(k).
	\end{gather*}
\end{proof}






















\end{document}