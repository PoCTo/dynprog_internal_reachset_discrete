\documentclass[10pt, a4paper]{article}
\usepackage{triada}

\graphicspath{{eps/}{png/}}

\renewcommand{\ell}{\mathcal{E}}

\usepackage{delim}

                           
\begin{document}
\thispagestyle{empty}

\begin{center}
\ \vspace{-3cm}

\includegraphics[width=0.5\textwidth]{msu.eps}\\
{\scshape Московский государственный университет имени М.~В.~Ломоносова}\\
Факультет вычислительной математики и кибернетики\\
Кафедра системного анализа

%\vspace{5cm}
\vfill
{\LARGE Отчет по практикуму}

\vspace{1cm}

{\Huge\bfseries <<Динамическое программирование \\ и процессы управления>>}
\end{center}

\vspace{2cm}

\begin{flushright}
  \large
  \textit{Студент 415 группы}\\
  Д.~И.~Степенский

  \vspace{5mm}

  \textit{Руководитель практикума}\\
  асс. Ю.~Ю.~Минаева
\end{flushright}

\vfill

\begin{center}
Москва, 2012
\end{center}

\newpage

\tableofcontents

\newpage

\section{Постановка задачи}
Рассматривается следующая дискретная система:
\begin{equation}\label{statement_system} \begin{cases}
{x}(k+1) = A(k)x(k)+B(k)u(k),\quad k_0\leqslant k\leqslant k_1],\\
x(k_0)\in \ell(x_0,X_0), \\
u(k) \in \ell(q(k),Q(k)),\\
u(k)\in \mathbb{R}^m, x(k)\in \mathbb{R}^n.
\end{cases} \end{equation}  

Требуется построить внутреннюю эллипсоидальную оценку множества достижимости данной системы.

\section{Теоретическая часть}
\subsection{Множество достижимости}
Вначале будем считать, что матрицы $B(i)Q(i)B(i)'$ и $A(i)$ несингулярны, т.е. имеют полный ранг. Случай сингулярных матриц будет разобран позднее.
\begin{df}
\textit{Эллипсоидом} $\ell(q,Q), Q\in\mathbb{R}^{n\times n}$ в $n$-мерном пространстве назовём множество точек
\[ \ell(q,Q) = \left\{ x\colon (x-q)Q^{-1}(x-q)' \leqslant 1 \right\}. \]
Здесь $q$ --- \textit{центр} эллипсоида, а $Q$ --- симметричная неотрицательно определенная \textit{матрица} эллипсоида. 
\end{df}
Эллипсоид назовем \textit{вырожденным}, если его матрица имеет неполный ранг, или, что равносильно, $\det Q = 0$.
\begin{df}
\textit{Множеством достижимости} дискретной системы, стартующей в момент $k_0$ из точки $x_0$ в момент времени $k$ называется следующее множество:
\[ \mathcal{X}(k,k_0,x_0) = \mathcal{X}[k] = \left\{ x \middle| \exists u(s), k_0\leqslant s\leqslant k\colon x(k|k_0,x_0)=x \right\}.\]
\end{df}
Понятие множества достижимости обобщается на случай произвольной стартовой точки из множества $\mathcal{X}_0$:
\[ \mathcal{X}(k,k_0,\mathcal{X}_0) = \left\{ x \middle| \exists u(s), k_0\leqslant s\leqslant k\colon x(k|k_0,x_0)\in \mathcal{X} \right\} = \bigcup\limits_{x_0\in\mathcal{X}_0}\mathcal{X}(k,k_0,x_0).\]
\begin{theorem}
Решением системы \eqref{statement_system} является 
\begin{equation}\label{solution}
	x(k) = \Phi(k,k_0)x_0 + \sum\limits_{i=k_0}^{k-1}\Phi(k,i+1)B(i)u(i),
\end{equation} 
где $\Phi(k,p)$ --- переходная матрица:
\begin{equation}\label{solution}
	\Phi(k+1,p) = A(k)\Phi(k,p),\ \Phi(p,p)=I.
\end{equation}
\end{theorem}
\begin{proof}
	Докажем по индукции. Для $k=k_0+1$ очевидно. Пусть для $k$ показано, тогда для $k+1$
	проверим подстановкой в \eqref{statement_system}:
	\begin{gather*}
		{x}(k+1) = \Phi(k+1,k_0)x_0 + \sum\limits_{i=k_0}^{k+1-1}\Phi(k+1,i+1)B(i)u(i) = 
		\\ = A(k)\Phi(k,k_0)x_0 + A(k)\sum\limits_{i=k_0}^{k+1-2}\Phi(k,i+1)B(i)u(i) + 
		\Phi(k+1,k+1)B(k)u(k) = \\ = A(k)x(k) + B(k)u(k).
	\end{gather*}
\end{proof}
\begin{imp}
	Используя теорему, множество достижимости системы \eqref{statement_system} можно записать следующим образом:
	\begin{gather}
		\notag \mathcal{X}(k,k_0,\ell(x_0,X_0)) = q_c(k) + \Phi(k,k_0)\ell(0,X_0) + 
			\sum\limits_{i=k_0}^{k-1}\Phi(k,i+1)B(i)\ell(0,Q(i)) = \\
		= q(k) + \Phi(k,k_0)\ell(0,X_0) + 
			\sum\limits_{i=k_0}^{k-1}\Phi(k,i+1)\ell(0,B(i)Q(i)B(i)'),
	\end{gather}
	где \begin{gather}\label{ellipsoids_center}
		q_c(k) = \Phi(k,k_0)x_0 + \sum\limits_{i=k_0}^{k-1}\Phi(k,i+1)B(i)q(i).
	\end{gather}
\end{imp}
Здесь знак <<$+$>> означает сумму множеств по Минковскому: $A+B= \left\{x+y \middle| x\in A,\, y\in B\right\}$, а умножение множества на матрицу производится по формуле $\Phi A = \left\{ \Phi x\middle| x\in A \right\}.$ Кроме того, использовалось свойство эллипса $\Phi\ell(x,X) = \ell(\Phi x,\Phi X\Phi')$.
\subsection{Внутренние эллипсоидальные оценки}
Если умножение на невырожденную матрицу не выводит из класса эллипсоидов, то сумма двух эллипсоидов, вообще говоря, может не принадлежать этому классу, поэтому множество достижимости конкретным эллипсоидом записать не получится. Однако возможно эллипсоидальное оценивание этого множества.

Для начала найдем опорную функцию множества достижимости, пользуясь свойством её линейности относительно сумм по Минковскому, а так же тем, что опорная функция эллипсоида $\sufu{l}{\ell(x,X)}$ равна $\scalar{l}{x}+\norm{l}_X$:
\begin{gather}\label{th:ellipsoids_follow}
\notag\sufu{l}{\mathcal{X}\left(k,k_0,\ell\left(x_0,X_0\right)\right)} = \sufu{l}{q_c\left(k\right)} + \sufu{l}{\Phi\left(k,k_0\right)\ell\left(0,X_0\right)} +\\ 
			\notag + 
			 \sum\limits_{i=k_0}^{k-1}\sufu{l}{\Phi\left(k,i+1\right)B\left(i\right)\ell\left(0,Q\left(i\right)\right)} = 
			 \scalar{l}{q_c\left(k\right)} + 
				\sufu{l}{\ell\left(0,\Phi\left(k,k_0\right)X_0\Phi\left(k,k_0\right)'\right)} + \\
			\notag +\sum\limits_{i=k_0}^{k-1} 
				\sufu{l}{
					\ell\left(0,\Phi\left(k,i+1\right)B\left(i\right)Q\left(i\right)B\left(i\right)'\Phi\left(k,i+1\right)'\right)} = 
			 \scalar{l}{q_c\left(k\right)} +\\+ \scalar{l}{\Phi\left(k,k_0\right)X_0\Phi\left(k,k_0\right)'l}^{1/2}
			  +
			\scalar{l}{\Phi\left(k,i+1\right)B\left(i\right)Q\left(i\right)B\left(i\right)'\Phi\left(k,i+1\right)'l}^{1/2}.
\end{gather}
Заметим, что если предположить, что центры всех эллипсов нулевые, то из \eqref{th:ellipsoids_follow} исчезнет лишь первое слагаемое.
\begin{theorem}\label{th_st:internal}
Эллипсоид, являющийся внутренней оценкой суммы $\ell(0,Q_1)+\dots+\ell(0,Q_n)$, имеет матрицу
\begin{equation}\label{th:ell} Q = Q_*'Q_*,\quad Q_*=\sum\limits_{1}^{n} S_kQ_k^{1/2},\ S_kS'_k=I, \end{equation}
причем для оптимальности (в смысле касания заданного множества в направлении $l$) необходима и достаточна линейная зависимость векторов $S_kQ_k^{1/2}l,\ k=1,\dots,n.$
\end{theorem}
\begin{proof}
Для доказательства внутренности оценки достаточно показать, что опорная функция эллипсоида с матрицей \eqref{th:ell} всюду не больше опорной функции суммы эллипсоидов. Сравним квадраты опорных функций, пользуясь неравенством Коши-Буняковского:
\begin{gather*}
	\sufu{l}{\ell(0,Q)}^2 = \scalar {l}{\left( \sum\limits_{k=1}^{n}S_k Q_k^{1/2} \right)'\left( \sum\limits_{k=1}^{n}S_k Q_k^{1/2} \right)l} = 
	\sum\limits_{k=1}^{n}\scalar{l}{Q_kl} +  \\
		+ 2\sum\limits_{k=1}^{n} \sum\limits_{p=k+1}^{n}\scalar{S_kQ_k^{1/2}l}{S_pQ_p^{1/2}l} 
	\leqslant \sum\limits_{k=1}^{n}\scalar{l}{Q_kl} + 
		2\sum\limits_{k=1}^{n} \sum\limits_{p=k+1}^{n}\scalar{l}{Q_kl}^{1/2}\scalar{l}{Q_pl}^{1/2} = \\
	= \left( \sum\limits_{k=1}^{n}\scalar{l}{Q_kl}^{1/2} \right)^2 
	= \sufu{l}{\sum\limits_{k=1}^{n} \ell\left(0,X_k\right)}^2.
\end{gather*}
Равенство в неравенстве Коши-Буняковского, то есть оптимальность в смысле касания, достигается тогда и только тогда, когда линейно зависимы векторы $S_kQ_k^{1/2}l$.
\end{proof}

Из описанного выше следует теорема об аппроксимации множества достижимости системы \eqref{statement_system}.
\begin{theorem}
Для каждого момента времени $k$
\[\mathcal{X}\left(k,k_0,\ell\left(x_0,X_0\right)\right) = 
	\bigcup\limits_{l}\ell\left( q_c(k), Q(l) \right), \]
	где $Q$ задается \eqref{th:ell}, $q_c(k)$ задается \eqref{ellipsoids_center}, и 
	\begin{equation}
		Q_*(l) = \left( \left(SX_0S'\right)^{1/2} + \sum\limits_{k_*=k_0}^{k-1}S_{k_*}\left(\Phi(k_0,k_*+1)B(k_*)Q(k_*)B(k_*)'\Phi(k_0,k_*+1)'\right)^{1/2} \right) 
		\Phi(k,k_0)'
	\end{equation}
\end{theorem}
\begin{proof}
	Так как для касания по направлению $l$ требуется линейная зависимость $S\left(\Phi\left(k,k_0\right)X_0\Phi\left(k,k_0\right)'\right)^{1/2}l$ и $S_{k_*}\left(\Phi\left(k,k_*+1\right)B\left(k_*\right)Q\left(k_*\right)B\left(k_*\right)'\Phi\left(k,k_*+1\right)'\right)^{1/2}l$  (где $S$ и  $S_{k_*}$ ортогональны) для всех $k_0 \leqslant k_* \leqslant k-1$, то по ортогональной $S$ (например, $S=I$) можно получить из выражения все нужные линейно зависимые векторы, и при таком определении они будут удовлетворять условию равенства в неравенстве Коши-Буняковского из теоремы \ref{th_st:internal}. Поэтому, вообще говоря, внутренние приближения зависят лишь от $l$.

Доказательство самой теоремы теперь проведем двусторонним вложением. Имеем вложенность правой части условия в левую, поскольку все оценки --- внутренние. Вложенность же левой в правую получается из следующих соображений: для произвольного $x \in \mathcal{X}\left(k,k_0,\ell\left(x_0,X_0\right)\right)$ рассмотрим вектор $l = x - q_c(k)$. Для такого $l$ существует внутренняя эллипсоидальная оценка, которая касается $\mathcal{X}\left(k,k_0,\ell\left(x_0,X_0\right)\right)$ по направлению $l$. Но тогда эта оценка содержит весь отрезок от $q_c(k)$ до точки касания, а точка $x$ лежит на этом отрезке, а значит, принадлежит и правой части.
\end{proof}
\begin{note}
Заметим, что везде выше использовалось лишь свойство корня симметричной матрицы $Q^{1/2}\left(Q'\right)^{1/2} = Q$, поэтому мы вправе (то есть выкладки не изменятся) определить квадратный корень из матрицы $S$, необязательно симметричной, как матрицу $S^{1/2}$ такую, что $S^{1/2}\left(S'\right)^{1/2} = S$. Вообще говоря, такой корень может быть не единственным, но это не имеет значения при данной задаче.
\end{note}
Теперь можно записать более краткий вид корней описанных в теореме матриц: 
\begin{gather}
	\label{root_formula1} S\left(\Phi\left(k,k_0\right)X_0\Phi\left(k,k_0\right)'\right)^{1/2}l =
		SX_0^{1/2}\Phi(k,k_0)'l, \\ 
	\notag S_{k_*}\left(\Phi\left(k,k_*+1\right)B\left(k_*\right)Q\left(k_*\right)B\left(k_*\right)'\Phi\left(k,k_*+1\right)'\right)^{1/2}l 
		=  S_{k_*}Q^{1/2}B(k_*)'\Phi\left(k,k_*+1\right)'l = \\
		\label{root_formula2} = \lambda_{k_*} SX_0^{1/2}\Phi(k,k_0)'l.
\end{gather}
Кроме того, можно формула для $\lambda$ будет иметь вид (получено из рассмотрения норм левой и правой части последнего равенства в \eqref{root_formula2}, учитывая ортогональность $S$ и $S_{k_*}$)
\begin{equation}\label{lambdas}
	\lambda_{k_*} = \frac {\scalar{l}{\Phi\left(k,k_*+1\right)B\left(k_*\right)Q\left(k_*\right)B\left(k_*\right)'\Phi\left(k,k_*+1\right)'l}^{1/2}}{\scalar{l}{\Phi\left(k,k_0\right)X_0\Phi\left(k,k_0\right)'l}^{1/2}}
\end{equation}
\begin{note}
Из последних формул видно, что при переходе $k$ к $k+1$ придётся пересчитывать значения $S_{k_*}$ и $\lambda_{k_*}$ для всех $k_*$, так как эти величины зависят от $k$, что значительно увеличивает объем вычислений. Для уменьшения количества операций рассмотрим параметризованное значение $l(k)$, удовлетворяющее 
	\begin{equation}\label{newl}
		l(k)=\Phi(k_0,k)'l(k_0) = \Phi(k_0,k)'l_0.
	\end{equation}
	Перепишем в этом случае \eqref{root_formula2} и \eqref{lambdas}:
	\begin{gather}\label{new_lambdas}
		S_{k_*}Q^{1/2}(k_*)B(k_*)'\Phi\left(k_0,k_*+1\right)'l = \lambda_{k_*}SX_0^{1/2}l_0, \\
		\lambda_{k_*} =
			 \frac{\scalar{l}{\Phi(k_0,k_*+1)B(k_*)Q(k_*)B(k_*)'\Phi\left(k_0,k_*+1\right)'l}^{1/2} }
			{\scalar{l}{X_0l}^{1/2} }
	\end{gather}	
	Заметим, что так как оператор $\Phi(k_0,k)$ невырожден, то если $l_0$ <<заметает>>  все возможные направления, то и $l(k)$ в любой момент времени обладает этим свойством.
\end{note}

\begin{theorem}
Для расчета при увеличении интервала времени можно использовать рекурсивную формулу
	\begin{gather}
		\notag Q(k+1) = Q(k) + \left(\Phi(k+1,k_0)B(k)Q(k)B(k)'\Phi(k+1,k_0)'\right) + \Phi(k+1,k_0)\cdot {}\\
		{} \cdot \left( M_k S_k Q^{1/2}(k)B(k)'\Phi(k_0,k+1)' + Q^{1/2}(k)B(k)'\Phi(k_0,k+1)'S_k'M_k' \right) \cdot \Phi(k+1,k_0)',
	\end{gather}
	причём
	\begin{gather}
		Q(k_0) = X_0,\\
		M_k = X_0^{1/2}S'+\sum\limits_{i=k_0}^{k-1}Q^{1/2}(i)B(i)'\Phi(k_0,i+1)'S_i'.
	\end{gather}
\end{theorem}
\begin{proof}
Данный результат получается при подстановке значений матриц эллипсоидов, составляющих множество достижимости в следующее равенство ($H_i$ --- матрицы некоторых эллипсоидов):
\begin{gather}
	Q(k+1) = \left(\sum\limits_{i=0}^{k+1}S_iH_i^{1/2}\right)'\left(\sum\limits_{i=0}^{k+1}S_iH_i^{1/2}\right) = \\
	= \left(\sum\limits_{i=0}^{k}S_iH_i^{1/2} + S_{k+1}H_{k+1}^{1/2} \right)'\left(\sum\limits_{i=0}^{k}S_iH_i^{1/2} + S_{k+1}H_{k+1}^{1/2}\right) = \\
	= Q(k) + H_{k+1} + \left(\sum\limits_{i=0}^{k}S_iH_i^{1/2}\right)' S_{k+1}H_{k+1}^{1/2} +
	\left(  S_{k+1}H_{k+1}^{1/2} \right)'\left(\sum\limits_{i=0}^{k}S_iH_i^{1/2}\right).
\end{gather}
\end{proof}



















\end{document}